\documentclass{report}

\usepackage{amsmath}
\usepackage{tikz}
\usepackage{verbatim}

\title{\Huge{\textbf{Proposal to \\Form a Coding Club}}}
\author{Ben Raz \texttt{<ben.raz2008@gmail.com>}}
\date{\today}

\begin{document}

    \maketitle

    \section{Introduction}

        \subsection{Disclaimer}

        Contained within this document is a detailed proposal for the creation of a coding club afterschool. While this document argues for the existence of a coding club, it does not specify when it shall occur.

        \subsection{Overview}

        This document will seek to overcome two thresholds, namely

        \begin{itemize}
            \item Need
            \item Interest in Student Population
        \end{itemize}

        While I firmly believe in the need and acceptance of a coding club, it's existence is at the sole discretion of the leadership team and/or other concerned parties not explicitly named.

    \section{The Need For A \textsc{Coding Club}}

        While many point to the \textsc{Digital Art Club} as a place to code, it's mandate is far wider and when coding comes up it is often for the purpose of making games using visual coding platforms like \textsc{Scratch}. Although coding comes up, it is often not for it's own sake and never for long. Most people in DA are not interested in learning to code.

        Aside from DA club, there are no other options for coding as of today, \today. The \textsc{Wednesday Art Club} used to be an option, but due to recent rule changes (another can of worms), coding is not allowed in any capacity to anyone, regardless of experience or skill level.

        Increasingly in the last few years, coding has begun to be taught to children at an early age, being seen as an easy-to-learn valuable skill. This makes the abscence of a coding club especially hard to take.

    \section{The Demand For A Coding Club}
    \section{Implementation Details}

        \subsection{Club Activities}



        \subsection{Techical Issues}

            The biggest issue to solve with a coding club is what computers to use and how. Chromebooks are out of the equation due to their limited set of features and third-party apps. While coding online with platforms like \textsc{Replit} is possible, it is not as convenient. Therefore, \textsc{MacBooks} are a better option. With \textsc{MacBooks}, it is possible to download apps and use the command line (Terminal). These \textsc{MacBooks} can be borrowed from the art room, as they are rarely used, if ever, during wednesday art club. This, however, does rule out the club occuring on Mondays.

            One obvious issue with this solution is the reality of computer restrictions. The reasons computer restrictions exist is to make sure that actions that modify the computer, like downloading apps or messing with the command line, are infeasable. A possible solution to this thorny issue is to lift restrictions on computers during \textsc{Coding Club}. This would create another issue with bad-actors.

            Bad actors are not in coding club to code, but to play games or otherwise abuse computers. This is where restrictions come in. But if restrictions have been lifted, the above behaviour would be too easy for some. While there are a few possible solutions to this problem, any solution would undeniably need to be made by those with more authority or knowledge of the issue, such as \textsc{Mr. Pantoliano} and/or the \textsc{Leadership Team}. For this above reason, this document will not weigh in on the issue any longer.

\end{document}


